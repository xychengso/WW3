\vsssub
\subsubsection{Spatial and temporal tracking of wave systems} \label{sec:ww3systrk}
\vsssub

\proddefH{ww3\_systrk}{w3systrk}{ww3\_systrk.ftn}
\proddeff{Input}{ww3\_systrk.inp}{Formatted input file for program.}{10}
\proddefa{partition.ww3}{Spectral partition file.}{11}
\proddefa{sys\_restart.ww3\opt}{Restart file with system memory.}{12}
\proddefa{sys\_mask.ww3\opt}{Mask file.}{13}
\proddeff{Output}{sys\_log.ww3}{Output log (appended with processor number in parallel run).}{20}
\proddefa{sys\_coord.ww3}{Lat/lon coordinates of fields.}{21}
\proddefa{sys\_hs.ww3}{Significant wave height fields of individual wave systems.}{22}
\proddefa{sys\_tp.ww3}{Peak period fields of individual wave systems.}{23}
\proddefa{sys\_dir.ww3}{Peak direction fields of individual wave systems}{24}
\proddefa{sys\_dspr.ww3}{Direction spread fields of individual wave systems.}{25}
\proddefa{sys\_pnt.ww3}{Point output file for significant wave height, peak period, and peak direction.}{26}
\proddefa{sys\_restart1.ww3}{Restart file.}{27}
\proddefa{*.nc}{NetCDF file.}{ }

\inpfile{ww3_systrk.tex}


\vspace{\baselineskip} 
\noindent 
Program currently implemented for regular grids only. The spatial and temporal
tracking is performed on the basis of the spectral partition data file. Both
the time interval and geographic domain over which wave systems are tracked
can be subsets of the data contained in the partition file. The combining
parameters {\code dirKnob} and {\code perKnob} are used to influence the
strictness of the system combining algorithm in geographic space, and {\code
  dirTimeKnob} and {\code perTimeKnob} are the corresponding parameters in
temporal space. Lower values imply stricter criteria, which results in
smaller, more numerous systems. This also typically increases the processing
time. Recommended values are given above. These values can be influenced
locally, for example around an island, by defining a mask file {\file
  sys\_mask.ww3}.  Parameters {\code hsKnob} and {\code wetPts} are a
low-energy and small system filters---all wave systems with an average
$H_\mathrm{m0}$ below {\code hsKnob} or with a size of less than {\code
  wetPts}*100\% of the overall domain size are purged. Parameters {\code
  seedLat} and {\code seedLon} influence the origin of the wave system search
spiral, with default at the center of model domain (indicated by {\code
  0. 0.}). At the end of a tracking run, the end state of system memory is
stored in {\file sys\_restart1.ww3}.  This file, renamed as {\file
  sys\_restart.ww3}, can be used to restart a tracking sequence from this
previous system memory state.

\pb
